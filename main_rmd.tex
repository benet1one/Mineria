% Options for packages loaded elsewhere
\PassOptionsToPackage{unicode}{hyperref}
\PassOptionsToPackage{hyphens}{url}
%
\documentclass[
]{article}
\usepackage{amsmath,amssymb}
\usepackage{iftex}
\ifPDFTeX
  \usepackage[T1]{fontenc}
  \usepackage[utf8]{inputenc}
  \usepackage{textcomp} % provide euro and other symbols
\else % if luatex or xetex
  \usepackage{unicode-math} % this also loads fontspec
  \defaultfontfeatures{Scale=MatchLowercase}
  \defaultfontfeatures[\rmfamily]{Ligatures=TeX,Scale=1}
\fi
\usepackage{lmodern}
\ifPDFTeX\else
  % xetex/luatex font selection
\fi
% Use upquote if available, for straight quotes in verbatim environments
\IfFileExists{upquote.sty}{\usepackage{upquote}}{}
\IfFileExists{microtype.sty}{% use microtype if available
  \usepackage[]{microtype}
  \UseMicrotypeSet[protrusion]{basicmath} % disable protrusion for tt fonts
}{}
\makeatletter
\@ifundefined{KOMAClassName}{% if non-KOMA class
  \IfFileExists{parskip.sty}{%
    \usepackage{parskip}
  }{% else
    \setlength{\parindent}{0pt}
    \setlength{\parskip}{6pt plus 2pt minus 1pt}}
}{% if KOMA class
  \KOMAoptions{parskip=half}}
\makeatother
\usepackage{xcolor}
\usepackage[margin=1in]{geometry}
\usepackage{graphicx}
\makeatletter
\def\maxwidth{\ifdim\Gin@nat@width>\linewidth\linewidth\else\Gin@nat@width\fi}
\def\maxheight{\ifdim\Gin@nat@height>\textheight\textheight\else\Gin@nat@height\fi}
\makeatother
% Scale images if necessary, so that they will not overflow the page
% margins by default, and it is still possible to overwrite the defaults
% using explicit options in \includegraphics[width, height, ...]{}
\setkeys{Gin}{width=\maxwidth,height=\maxheight,keepaspectratio}
% Set default figure placement to htbp
\makeatletter
\def\fps@figure{htbp}
\makeatother
\setlength{\emergencystretch}{3em} % prevent overfull lines
\providecommand{\tightlist}{%
  \setlength{\itemsep}{0pt}\setlength{\parskip}{0pt}}
\setcounter{secnumdepth}{-\maxdimen} % remove section numbering
\usepackage{graphicx}
\usepackage{fancyhdr}
\pagestyle{fancy}
\fancyhead[R]{}
\fancyhead[L]{}
\fancyhead[C]{\includegraphics[height=1.5cm]{Logos.png}}
\fancyfoot[R]{}
\fancyfoot[C]{\thepage}
\renewcommand{\headrulewidth}{0.7pt}
\setlength{\headheight}{48pt}
\addtolength{\topmargin}{-25pt}
\usepackage{titling}
\usepackage{graphicx}
\pretitle{\vspace{3cm}\begin{center}\LARGE}
\posttitle{\vspace{0.5cm}\end{center}}
\preauthor{\begin{center}\large}
\postauthor{\end{center}}
\predate{\begin{center}}
\postdate{\vspace{3cm}\end{center}\begin{center}\includegraphics[width=8cm]{Logos.png}\end{center}}
\ifLuaTeX
  \usepackage{selnolig}  % disable illegal ligatures
\fi
\usepackage{bookmark}
\IfFileExists{xurl.sty}{\usepackage{xurl}}{} % add URL line breaks if available
\urlstyle{same}
\hypersetup{
  pdftitle={Song Popularity Prediction},
  hidelinks,
  pdfcreator={LaTeX via pandoc}}

\title{Song Popularity Prediction}
\usepackage{etoolbox}
\makeatletter
\providecommand{\subtitle}[1]{% add subtitle to \maketitle
  \apptocmd{\@title}{\par {\large #1 \par}}{}{}
}
\makeatother
\subtitle{An Analysis Using Data Mining Techniques}
\author{Garcia Garcia, Bernat\\
Maria Montés, Iker\\
Rota, Davide\\
Tobella Jacomet, Pol}
\date{2025-10-02}

\begin{document}
\maketitle

\newpage
\tableofcontents
\newpage

\section{Introduction}\label{introduction}

In this paper we will analyze data related to spotify tracks, with the
objective of determining a way to predict a song/track popularity in the
site through information related to that track. \textbf{\emph{Needs to
be expanded a bit probably}}

\section{Motivation}\label{motivation}

Our objective is to develop a regression model capable of predicting the
popularity of songs using musical attributes and associated metadata.
This project aims to identify patterns and characteristics that explain
why certain songs achieve higher levels of popularity, enabling
streaming platforms to make more informed decisions regarding
recommendations and playlist curation.

Specifically, the challenge consists of building a model that minimizes
the Mean Absolute Percentage Error (MAPE), providing accurate
predictions of each song's popularity (song\_popularity). Additionally,
we aim to analyze which musical attributes have the greatest influence
on song popularity, in order to extract valuable insights for marketing
and promotion strategies in the music industry.

In addition to the main objective, we seek to evaluate the quality and
consistency of the provided data to ensure that the information used for
modeling is reliable. An exploratory data analysis (EDA) will be
conducted to select the most relevant variables and detect potential
issues, such as missing values or outliers. Finally, we aim to propose
recommendations for improving future predictions, either through the
incorporation of new variables or through adjustments in the modeling
methodology.

To assess model performance, we will primarily use control metrics such
as MAPE, and complementary metrics such as RMSE and MAE. In addition,
business-oriented indicators will be considered, such as the ability to
identify potential hit songs before they achieve success on streaming
platforms.

External Supporting sources: - Kaggle competition:
\url{https://www.kaggle.com/competitions/prediccion-de-la-popularidad-de-canciones/overview}
- Spotify API Documentation about musical attributes:
\url{https://developer.spotify.com/documentation/web-api}

\newpage

\section{Metadata}\label{metadata}

Our dataset consists of 13186 imputs of spotify tracks with 15
variables. From those, 3 are qualitative variables while 12 are
quantitative variables.

\textbf{\emph{Probably some more explanation needed}}

\subsection{Data source}\label{data-source}

The data has been obtained from the file \emph{``train.csv''}. This file
contains information related to tracks of Spotify, openly shared on the
net. The data originally comes from the platform Kaggle, specifically
from the contest \emph{Predicción de la Popularidad de Canciones}.

\emph{Disclaimer: This data will not be used to train machine learning
or AI models, as per the Policy Note of Spotify.}

\newpage

\section{Overview of the Data}\label{overview-of-the-data}

\subsection{Qualitative variables}\label{qualitative-variables}

· \textbf{Audio mode}: Mode indicates the modality (\emph{major or
minor}) of a track, the type of scale from which its melodic content is
derived. Major is represented by 1 and minor is 0. Example: 0

· \textbf{Key}: The key the track is in. Integers map to pitches using
standard Pitch Class notation. E.g. 0 = C, 1 = C♯/D♭, 2 = D, and so on.
If no key was detected, the value is -1. Range: -1..11 Example: 9

· \textbf{Time signature}: An estimated time signature. The time
signature (meter) is a notational convention to specify how many beats
are in each bar (or measure). The time signature ranges from 3 to 7
indicating time signatures of ``3/4'', to ``7/4''. Range: 3 - 7 Example:
4

\subsection{Quantitative variables}\label{quantitative-variables}

\subsubsection{Relative/Ratio variables
(0..1)}\label{relativeratio-variables-0..1}

· \textbf{Liveness}: Detects the presence of an audience in the
recording. Higher liveness values represent an increased probability
that the track was performed live. \emph{A value above 0.8 provides
strong likelihood that the track is live.} Example: 0.0866

· \textbf{Danceability}: Danceability describes how suitable a track is
for dancing \emph{based on a combination of musical elements including
tempo, rhythm stability, beat strength, and overall regularity}. A value
of 0.0 is least danceable and 1.0 is most danceable. Example: 0.585

· \textbf{Audio valence}: A measure from 0.0 to 1.0 describing the
musical positiveness conveyed by a track. \emph{Tracks with high valence
sound more positive} (e.g.~happy, cheerful, euphoric), while tracks with
low valence sound more negative (e.g.~sad, depressed, angry). Example:
0.428

· \textbf{Energy}: Energy is a measure from 0.0 to 1.0 and represents a
perceptual \emph{measure of intensity and activity}. Typically,
energetic tracks feel \emph{fast, loud, and noisy}. For example, death
metal has high energy, while a Bach prelude scores low on the scale.
Perceptual features contributing to this attribute include dynamic
range, perceived loudness, timbre, onset rate, and general entropy.
Example: 0.842

· \textbf{Acousticness}: A confidence measure from 0.0 to 1.0 of whether
the track is acoustic. \emph{1.0 represents high confidence the track is
acoustic}. Example: 0.00242

· \textbf{Speechiness}: Speechiness detects the presence of \emph{spoken
words in a track}. The more exclusively speech-like the recording
(e.g.~talk show, audio book, poetry), the closer to 1.0 the attribute
value. \emph{Values above 0.66 describe tracks that are probably made
entirely of spoken words}. Values between 0.33 and 0.66 describe tracks
that may contain both music and speech, either in sections or layered,
including such cases as rap music. Values below 0.33 most likely
represent music and other non-speech-like tracks. Example: 0.0556

· \textbf{Instrumentalness}: Predicts whether a track contains no
vocals. ``Ooh'' and ``aah'' sounds are treated as instrumental in this
context. Rap or spoken word tracks are clearly ``vocal''. \emph{The
closer the instrumentalness value is to 1.0, the greater likelihood the
track contains no vocal content}. Values above 0.5 are intended to
represent instrumental tracks, but confidence is higher as the value
approaches 1.0. Example: 0.00686

· \textbf{Song popularity}: Objective variable, which describes the
popularity of the song. The popularity of a track is a value between 0
and 100, with 100 being the most popular. Example: 69

\newpage

\subsubsection{Absolute variables}\label{absolute-variables}

· \textbf{ID}: A count, from 1 to 13186 \textbf{\emph{Not spotify ID}}

· \textbf{Loudness}: The overall loudness of a track in decibels (dB).
Loudness values are averaged across the entire track and are useful for
comparing relative loudness of tracks. Loudness is the quality of a
sound that is the primary psychological correlate of physical strength
(amplitude). \emph{Values typically range between -60 and 0 db}.
Example: -5.883

· \textbf{Song duration (ms)}: The duration of the track in
milliseconds. Example: 237040

· \textbf{Tempo}: The overall estimated tempo of a track in beats per
minute (BPM). In musical terminology, tempo is the speed or pace of a
given piece and derives directly from the average beat duration.
Example: 118.211

\subsection{The data}\label{the-data}

In this section we show a summary of our data

\begin{verbatim}
##        ID           liveness        loudness        danceability  
##  Min.   :    1   Min.   :0.015   Min.   :-36.729   Min.   :0.000  
##  1st Qu.: 3297   1st Qu.:0.093   1st Qu.: -9.095   1st Qu.:0.531  
##  Median : 6594   Median :0.121   Median : -6.587   Median :0.645  
##  Mean   : 6594   Mean   :0.179   Mean   : -7.449   Mean   :0.634  
##  3rd Qu.: 9890   3rd Qu.:0.221   3rd Qu.: -4.921   3rd Qu.:0.749  
##  Max.   :13186   Max.   :0.986   Max.   :  1.342   Max.   :0.987  
##                  NA's   :3956    NA's   :3956      NA's   :3956   
##  song_duration_ms  time_signature  audio_valence       energy     
##  Min.   :  26186   Min.   :0.000   Min.   :0.000   Min.   :0.002  
##  1st Qu.: 184000   1st Qu.:4.000   1st Qu.:0.339   1st Qu.:0.510  
##  Median : 211233   Median :4.000   Median :0.530   Median :0.670  
##  Mean   : 217955   Mean   :3.957   Mean   :0.531   Mean   :0.645  
##  3rd Qu.: 242292   3rd Qu.:4.000   3rd Qu.:0.728   3rd Qu.:0.817  
##  Max.   :1799346   Max.   :5.000   Max.   :0.982   Max.   :0.999  
##  NA's   :3956      NA's   :3956    NA's   :3956    NA's   :3956   
##      tempo         acousticness    speechiness         key        
##  Min.   :  0.00   Min.   :0.000   Min.   :0.000   Min.   : 0.000  
##  1st Qu.: 99.01   1st Qu.:0.024   1st Qu.:0.038   1st Qu.: 2.000  
##  Median :120.27   Median :0.131   Median :0.056   Median : 5.000  
##  Mean   :121.68   Mean   :0.258   Mean   :0.103   Mean   : 5.306  
##  3rd Qu.:140.00   3rd Qu.:0.422   3rd Qu.:0.122   3rd Qu.: 8.000  
##  Max.   :242.32   Max.   :0.996   Max.   :0.940   Max.   :11.000  
##  NA's   :3956     NA's   :3956    NA's   :3956    NA's   :3956    
##  instrumentalness   audio_mode    song_popularity 
##  Min.   :0.000    Min.   :0.000   Min.   :  0.00  
##  1st Qu.:0.000    1st Qu.:0.000   1st Qu.: 40.00  
##  Median :0.000    Median :1.000   Median : 56.00  
##  Mean   :0.075    Mean   :0.624   Mean   : 52.94  
##  3rd Qu.:0.002    3rd Qu.:1.000   3rd Qu.: 69.00  
##  Max.   :0.997    Max.   :1.000   Max.   :100.00  
##  NA's   :3956     NA's   :3956
\end{verbatim}

\end{document}
